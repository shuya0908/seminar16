\documentclass{jsarticle}
% \documentclass[b4paper,landscape,14pt]{jsarticle}
\title{}
\author{}
\date{
% \number\year 年 \number\month 月
}
\usepackage{fenrir_v1_4_0}
\usepackage{ethm_v1_1_0}
\mathtoolsset{showonlyrefs=true}

\begin{document}
% \maketitle
\setcounter{section}{7}
\section{Stochastic Differential Equations}
\subsection{Motivation and General Definitions}

\begin{screen}
    \begin{thm*}[Yamada-Watanabe]
        $E(\sigma, b)$ に関して weak existence, pathwise uniqueness 成立.
        このとき
        \begin{enumerate}[label=(\roman*)]
            \item
            $E(\sigma, b)$ に関する weak uniqueness 成立.
            \item
            $\forall (\Omega, \mathcal{F}, (\mathcal{F}_t), P)$: filtered prob. sp., $\forall B$: $(\mathcal{F}_t)$-BM, $\forall x\in\mathbf{R}^d, \exists! X$: $E_x(\sigma, b)$ の(一意な)strong solution.
        \end{enumerate}
    \end{thm*}
\end{screen}

この結果は Lipschitz 連続性が成り立っているとき,直接従う.

\subsection{The Lipschitz Case}

この節では,係数 $\sigma, b$ に対して以下の条件を仮定する:
\begin{screen}
    \textbf{\underline{仮定} }
    $\sigma, b$: $\mathbf{R_+}\times\mathbf{R}^d$ 上連続かつ変数 $x$ に関して Lipschitz 連続,つまり $\exists K\ge0$ s.t. $\forall t\ge0, \forall x\in\mathbf{R}^d, \forall y\in\mathbf{R}^d,$
    \begin{align}
        \left\lvert\sigma(t, x)-\sigma(t, y)\right\rvert
        &\le K\left\lvert x-y\right\rvert, \\
        \left\lvert b(t, x)-b(t, y)\right\rvert
        &\le K\left\lvert x-y\right\rvert.
    \end{align}
\end{screen}

ここで,後述の Theorem \ref{thm:803} の証明内で用いる補題を先に紹介する.

\setcounter{thm}{3}

\begin{screen}
    \begin{lem}[Gronwall's lemma]\label{lem:804}~
        \begin{itemize}
            \item 
            $T>0$
            \item 
            $g:[0, T]\to[0, \infty)$: 有界可測関数
            \item 
            $\exists a\ge0, \exists b\ge0$ s.t. $\forall t\in[0, T], $
            $$
            g(t)\le a+b\int_0^t g(s)ds
            $$
        \end{itemize}
        $\implies \forall t\in[0, T], $
        $$
        g(t)\le ae^{bt}.
        $$
    \end{lem}
\end{screen}

\begin{proof}
    $\forall n\ge2$ に対し
    $$
    \int_0^{t} ds_1\int_0^{s_1} ds_2\dotsb\int_0^{s_{n-1}} ds_n
    = \frac{t^n}{n!}
    $$
    が成り立つことを用いる.
    $\forall t\in[0, T]$ に対し
    \begin{align}
        g(t)
        &\le a+b\int_0^t ds_1g(s_1) \\
        &\le a+b\int_0^t ds_1(a+b\int_0^{s_1} ds_2g(s_2)) \\
        &= a
        + ab\UB{\int_0^t ds_1}{=t}
        + b^2\int_0^t ds_1\int_0^{s_1} ds_2g(s_2) \\
        &\le a
        + abt
        + b^2\int_0^t ds_1\int_0^{s_1} ds_2(a+b\int_0^{s_2} ds_3g(s_3)) \\
        &= a
        + abt
        + ab^2\UB{\int_0^t ds_1\int_0^{s_1} ds_2}{=t^2/2}
        + b^3\int_0^t ds_1\int_0^{s_1} ds_2\int_0^{s_2} ds_3g(s_3) \\
        &\le \dotsb \\
        &\le a+a(bt)+a\frac{(bt)^2}{2}+\dotsb+a\frac{(bt)^n}{n!}
        + b^{n+1}\int_0^t ds_1\int_0^{s_1} ds_2\dotsb\int_0^{s_n} ds_{n+1}\UB{g(s_{n+1})}{\le\exists A\ge0\text{($g$: bdd. on $[0, T]$ より)}} \\
        &\le a\UB{\sum_{k=0}^{n}\frac{(bt)^k}{k!}}{\to e^{bt}\text{ as }n\to\infty}
        + A\UB{\frac{(bt)^{n+1}}{(n+1)!}}{\to0\text{ as }n\to\infty}
        \xrightarrow{n\to\infty}ae^{bt}.
    \end{align}
\end{proof}

\setcounter{thm}{2}

\begin{screen}
    \begin{thm}\label{thm:803}
        上の仮定の下で
        \begin{enumerate}[label=(\roman*)]
            \item
            $E(\sigma, b)$ に関して pathwise uniqueness が成立.
            \item
            $\forall (\Omega, \mathcal{F}, (\mathcal{F}_t), P)$: filtered prob. sp., $\forall B$: $(\mathcal{F}_t)$-BM, $\forall x\in\mathbf{R}^d, \exists! X$: $E_x(\sigma, b)$ の(一意な)strong solution.
        \end{enumerate}
    \end{thm}
\end{screen}

$E(\sigma, b)$ に関して 
\begin{itemize}
    \item 
    Theorem \ref{thm:803}
    $\implies $ weak existence 成立
    \item
    Theorem 8.5
    $\implies $ weak uniqueness 成立
    
    (weak existence と pathwise uniqueness が成立していることから,Yamada-Watanabe の定理を用いても示せる)
\end{itemize}

\begin{remark*}
    $\sigma, b$ における Lipschitz 条件は「局所化」できる.
    つまり定数 $K$ が時間パラメータ $t$ と空間変数 $x, y$ が考慮されるコンパクト集合に依存することがある.
    しかしながら,局所 Lipschitz 条件は解の大域的存在,つまり任意の $t$ に対し定義された解の存在を保証するのに十分ではない(解が「爆発」する,つまり有限時間内に解の大きさが無限大に発散することがある).

    \begin{ex*}
        $$
        X_t
        = 1+\int_0^t X_s^2 ds
        $$
        の一意解は $X_t=\frac{1}{1-t}\ (0\le t<1)$ であるが,これは $t\uparrow1$ のとき爆発する.
    \end{ex*}
    
    大域的な解の存在の結果を得るためには,1 次増大条件
    $$
    \left\lvert \sigma(t, x)\right\rvert\le K(1+\left\lvert x\right\rvert), \quad
    \left\lvert b(t, x)\right\rvert\le K(1+\left\lvert x\right\rvert)
    $$
    を課す必要がある.
\end{remark*}

\end{document}
