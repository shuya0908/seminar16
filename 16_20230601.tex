\documentclass{jsarticle}
% \documentclass[b4paper,landscape,14pt]{jsarticle}
\title{}
\author{}
\date{
% \number\year 年 \number\month 月
}
\usepackage{fenrir_v1_4_0}
\usepackage{ethm_v1_1_0}
\mathtoolsset{showonlyrefs=true}

\begin{document}
% \maketitle
\setcounter{section}{7}
\section{Stochastic Differential Equations}
\subsection{Motivation and General Definitions}

\begin{screen}
    \begin{thm*}[Yamada-Watanabe]
        $E(\sigma, b)$ に関して weak existence, pathwise uniqueness 成立.
        このとき
        \begin{enumerate}[label=(\roman*)]
            \item
            $E(\sigma, b)$ に関する weak uniqueness 成立.
            \item
            $\forall (\Omega, \mathcal{F}, (\mathcal{F}_t), P)$: filtered prob. sp., $\forall B$: $(\mathcal{F}_t)$-BM, $\forall x\in\mathbf{R}^d, \exists! X$: $E_x(\sigma, b)$ のstrong solution.
        \end{enumerate}
    \end{thm*}
\end{screen}

この結果は Lipschitz 連続性が成り立っているとき,直接従う.

\subsection{The Lipschitz Case}

この節では,係数 $\sigma, b$ に対して以下の条件を仮定する:
\begin{screen}
    \textbf{\underline{仮定} }
    $\sigma, b$: $\mathbf{R_+}\times\mathbf{R}^d$ 上連続かつ変数 $x$ に関して Lipschitz 連続,つまり $\exists K\ge0$ s.t. $\forall t\ge0, \forall x\in\mathbf{R}^d, \forall y\in\mathbf{R}^d,$
    \begin{align}
        \left\lvert\sigma(t, x)-\sigma(t, y)\right\rvert
        &\le K\left\lvert x-y\right\rvert, \\
        \left\lvert b(t, x)-b(t, y)\right\rvert
        &\le K\left\lvert x-y\right\rvert.
    \end{align}
\end{screen}

ここで,後述の Theorem \ref{thm:803} の証明内で用いる補題を先に紹介する.

\setcounter{thm}{3}

\begin{screen}
    \begin{lem}[Gronwall's lemma]\label{lem:804}~
        \begin{itemize}
            \item 
            $T>0$
            \item 
            $g:[0, T]\to[0, \infty)$: 有界可測関数
            \item 
            $\exists a\ge0, \exists b\ge0$ s.t. $\forall t\in[0, T], $
            $$
            g(t)\le a+b\int_0^t g(s)ds
            $$
        \end{itemize}
        $\implies \forall t\in[0, T], $
        $$
        g(t)\le ae^{bt}.
        $$
    \end{lem}
\end{screen}

\begin{proof}
    $\forall n\ge2$ に対し
    $$
    \int_0^{t} ds_1\int_0^{s_1} ds_2\dotsb\int_0^{s_{n-1}} ds_n
    = \frac{t^n}{n!}
    $$
    が成り立つことを用いる.
    $\forall t\in[0, T]$ に対し
    \begin{align}
        g(t)
        &\le a+b\int_0^t ds_1g(s_1) \\
        &\le a+b\int_0^t ds_1(a+b\int_0^{s_1} ds_2g(s_2)) \\
        &= a
        + ab\UB{\int_0^t ds_1}{=t}
        + b^2\int_0^t ds_1\int_0^{s_1} ds_2g(s_2) \\
        &\le a
        + abt
        + b^2\int_0^t ds_1\int_0^{s_1} ds_2(a+b\int_0^{s_2} ds_3g(s_3)) \\
        &= a
        + abt
        + ab^2\UB{\int_0^t ds_1\int_0^{s_1} ds_2}{=t^2/2}
        + b^3\int_0^t ds_1\int_0^{s_1} ds_2\int_0^{s_2} ds_3g(s_3) \\
        &\le \dotsb \\
        &\le a+a(bt)+a\frac{(bt)^2}{2}+\dotsb+a\frac{(bt)^n}{n!}
        + b^{n+1}\int_0^t ds_1\int_0^{s_1} ds_2\dotsb\int_0^{s_n} ds_{n+1}\UB{g(s_{n+1})}{\le\exists A(\ge0)\text{($g$: bdd. on $[0, T]$ より)}} \\
        &\le a\UB{\sum_{k=0}^{n}\frac{(bt)^k}{k!}}{\to e^{bt}\text{ as }n\to\infty}
        + A\UB{\frac{(bt)^{n+1}}{(n+1)!}}{\to0\text{ as }n\to\infty}
        \xrightarrow{n\to\infty}ae^{bt}.
    \end{align}
\end{proof}

\setcounter{thm}{2}

\begin{screen}
    \begin{thm}\label{thm:803}
        上の仮定の下で
        \begin{enumerate}[label=(\roman*)]
            \item
            $E(\sigma, b)$ に関して pathwise uniqueness が成立.
            \item
            $\forall (\Omega, \mathcal{F}, (\mathcal{F}_t), P)$: filtered prob. sp., $\forall B$: $(\mathcal{F}_t)$-BM, $\forall x\in\mathbf{R}^d, \exists! X$: $E_x(\sigma, b)$ の(一意な)strong solution.
        \end{enumerate}
    \end{thm}
\end{screen}

$E(\sigma, b)$ に関して 
\begin{itemize}
    \item 
    Theorem \ref{thm:803}
    $\implies $ weak existence 成立
    \item
    Theorem 8.5
    $\implies $ weak uniqueness 成立
    
    (weak existence と pathwise uniqueness が成立していることから,Yamada-Watanabe の定理を用いても示せる)
\end{itemize}

\begin{remark*}
    $\sigma, b$ における Lipschitz 条件は「局所化」できる.
    つまり定数 $K$ が時間パラメータ $t$ と空間変数 $x, y$ が考慮されるコンパクト集合に依存することがある.
    しかしながら,局所 Lipschitz 条件は解の大域的存在,つまり任意の $t$ に対し定義された解の存在を保証するのに十分ではない(解が「爆発」する,つまり有限時間内に解の大きさが無限大に発散することがある).

    \begin{ex*}
        $$
        X_t
        = 1+\int_0^t X_s^2 ds
        $$
        の一意解は $X_t=\frac{1}{1-t}\ (0\le t<1)$ であるが,これは $t\uparrow1$ のとき爆発する.
    \end{ex*}
    
    大域的な解の存在の結果を得るためには,1 次増大条件
    $$
    \left\lvert \sigma(t, x)\right\rvert\le K(1+\left\lvert x\right\rvert), \quad
    \left\lvert b(t, x)\right\rvert\le K(1+\left\lvert x\right\rvert)
    $$
    を課す必要がある.
\end{remark*}

\begin{proof}
    議論を単純にするため,$d=m=1$ の場合を考える.
    \begin{enumerate}[label=(\roman*)]
        \item
        (同一の filtered prob sp. 上,同一の BM に対して定義された)$X_0=X'_0$ を満たす 2 つの解 $X, X'$ を考える.
        
        $M>0$: 固定
        $$
        \tau_M:=\inf\{t\ge0:\left\lvert X_t\right\rvert\vee\left\lvert X'_t\right\rvert\ge M\}
        $$
        と定めると $\forall t\ge0$ に対し
        \begin{align}
            X_{t\wedge\tau_M}
            &= X_0
            + \int_0^{t\wedge\tau_M}\sigma(s, X_s)dB_s
            + \int_0^{t\wedge\tau_M}b(s, X_s)ds, \\
            X'_{t\wedge\tau_M}
            &= X'_0
            + \int_0^{t\wedge\tau_M}\sigma(s, X'_s)dB_s
            + \int_0^{t\wedge\tau_M}b(s, X'_s)ds.
        \end{align}

        $T>0$: 固定

        $\forall t\in[0, T]$ に対し
        \begin{align}
            E[(X_{t\wedge\tau_M}-X'_{t\wedge\tau_M})^2]
            &= E[(\int_0^{t\wedge\tau_M}(\sigma(s, X_s)-\sigma(s, X'_s))dB_s
            + \int_0^{t\wedge\tau_M}(b(s, X_s)-b(s, X'_s))ds)^2] \\
            &\le 
            \begin{multlined}[t]
                2(E[(\int_0^{t\wedge\tau_M}(\sigma(s, X_s)-\sigma(s, X'_s))dB_s)^2] \\
                + E[(\int_0^{t\wedge\tau_M}(b(s, X_s)-b(s, X'_s))ds)^2])\quad
                (\because (a+b)^2\le 2(a^2+b^2))
            \end{multlined} \\
            &\stackrel{\text{(5.14), Schwarz}}{\le}
            \begin{multlined}[t]
                2(E[\int_0^{t\wedge\tau_M}(\sigma(s, X_s)-\sigma(s, X'_s))^2 ds] \\
                + TE[\int_0^{t\wedge\tau_M}(b(s, X_s)-b(s, X'_s))^2 ds])
            \end{multlined} \\
            &\stackrel{\text{Lipschitz}}{\le}
            2(K^2 E[\int_0^{t\wedge\tau_M}(X_s-X'_s)^2 ds]
            + K^2 TE[\int_0^{t\wedge\tau_M}(X_s-X'_s)^2 ds]) \\
            &= 2K^2(1+T)E[\int_0^{t\wedge\tau_M}(X_s-X'_s)^2 ds] \\
            &\le \UB{2K^2(1+T)}{=:C}E[\int_0^t(X_{s\wedge\tau_M}-X'_{s\wedge\tau_M})^2 ds]\nazo \\
            &\beq{\text{Fubini}} C\int_0^t E[(X_{s\wedge\tau_M}-X'_{s\wedge\tau_M})^2] ds.
        \end{align}

        よって関数 $[0, T]\ni t\mapsto E[(X_{t\wedge\tau_M}-X'_{t\wedge\tau_M})^2]\in[0, \infty)$ を $h$ と定めると,$\forall t\in[0, T]$ に対し
        $$
        h(t)
        \le C\int_0^t h(s)ds.
        $$
        
        ここで $\tau_M$ の定め方から $\left\lvert X_{t\wedge\tau_M}\right\rvert\vee\left\lvert X'_{t\wedge\tau_M}\right\rvert\le M,$ ゆえに $(X_{t\wedge\tau_M})^2, (X'_{t\wedge\tau_M})^2\le M^2$ が言えるので
                \begin{align}
                    h(t)
                    &= E[(X_{t\wedge\tau_M}-X'_{t\wedge\tau_M})^2] \\
                    &\le 2(E[(X_{t\wedge\tau_M})^2]+E[(X'_{t\wedge\tau_M})^2]) \le 4M^2.
                \end{align}
        したがって $h$ は有界なので Lemma \ref{lem:804} が適用でき,$h(t)\le0\cdot e^{Ct}=0$ より $h\equiv0.$
        ゆえに $X, X'$ の連続性と合わせて $P(\forall t\in[0, T], X_{t\wedge\tau_M}=X'_{t\wedge\tau_M})=1$ が成り立つ.
        また a.s. で $\lim_{M\to\infty}\tau_M=\infty$ となるので $P(\forall t\in[0, T], X_t=X'_t)=1.$
        よって pathwise uniqueness が成り立つことが言えた.
        
        \item
        Picard の逐次近似法を用いて解を構成する.
        求めたい解の逐次近似列を,帰納的に
        \begin{align}
            X_t^{(0)}
            &:= x, \\
            X_t^{(1)}
            &:= x
            + \int_0^t \sigma(s, x)dB_s
            + \int_0^t b(s, x)ds, \\
            &\ \ \vdots \\
            X_t^{(n)}
            &:= x
            + \int_0^t \sigma(s, X_s^{(n-1)})dB_s
            + \int_0^t b(s, X_s^{(n-1)})ds
        \end{align}
        と定める($\forall n\ge1$ に対し $X^{(n)}$ が連続かつ $(\mathcal{F}_t)$-adapted であることが帰納的にわかるため,右辺の確率積分は well-defined に定まる).

        $T>0$: 固定

        $\forall n\ge1, \forall t\in[0, T]$ に対し
        $$
        g_n(t)
        := E[\sup_{s\in[0, t]}\left\lvert X_s^{(n)}-X_s^{(n-1)}\right\rvert^2]
        $$
        と定める(この関数列 $\{g_n\}_{n\ge1}$ が有界であることを帰納的に確かめる).
        $\sigma(\cdot, x), b(\cdot, x)$: $[0, T]$ 上連続 $\implies [0, T]$ 上有界,つまり
        $$
        \exists K_T\ge0\text{ s.t. }\sup_{[0, T]}\left\lvert \sigma(\cdot, x)\right\rvert\vee\sup_{[0, T]}\left\lvert b(\cdot, x)\right\rvert\le K_T
        $$
        なので $\forall t\in[0, T]$ に対し
        \begin{align}
            g_1(t)
            &= E[\sup_{s\in[0, t]}\left\lvert X_s^{(1)}-X_s^{(0)}\right\rvert^2] \\
            &= E[\sup_{s\in[0, t]}\left\lvert \int_0^s \sigma(u, x)dB_u
            + \int_0^s b(u, x)du\right\rvert^2] \\
            &\le 2(E[\sup_{s\in[0, t]}(\int_0^s \sigma(u, x)dB_u)^2]
            + E[\sup_{s\in[0, t]}(\int_0^s b(u, x)du)^2]) \\
            &\stackrel{\text{Doob}}{\le} 2(4E[(\int_0^t \sigma(u, x)dB_u)^2]
            + \sup_{s\in[0, t]}sE[\int_0^s (b(u, x)^2)du]) \\
            &\le 2(4E[\int_0^T (\sigma(u, x))^2du]
            + TE[\int_0^T (b(u, x))^2du]) \\
            &\le 2K_T^2T(4+T)=:C'_T.
        \end{align}

        ここで
        \begin{align}
            \left\lvert X_t^{(n)}-X_t^{(n-1)}\right\rvert^2
            &= \left\lvert \int_0^t (\sigma(s, X_s^{(n)})-\sigma(s, X_s^{(n-1)}))dB_s
            + \int_0^t (b(s, X_s^{(n)})-b(s, X_s^{(n-1)}))ds\right\rvert^2 \\
            &\le 2((\int_0^t (\sigma(s, X_s^{(n)})-\sigma(s, X_s^{(n-1)}))dB_s)^2
            + (\int_0^t (b(s, X_s^{(n)})-b(s, X_s^{(n-1)}))ds)^2)
        \end{align}
        が成り立つことより,
        \begin{align}
            g_{n+1}(t)
            &= E[\sup_{s\in[0, t]}\left\lvert X_s^{(n+1)}-X_s^{(n)}\right\rvert^2] \\
            &\le 
            \begin{multlined}[t]
                2(E[\sup_{s\in[0, t]}(\int_0^s (\sigma(u, X_u^{(n+1)})-\sigma(u, X_u^{(n)}))dB_u)^2] \\
                + E[\sup_{s\in[0, t]}(\int_0^s (b(u, X_u^{(n+1)})-b(u, X_u^{(n)}))du)^2])
            \end{multlined} \\
            &\stackrel{\text{BDG}}{\le}
            \begin{multlined}[t]
                2(C_{(2)}E[\int_0^t (\sigma(u, X_u^{(n+1)})-\sigma(u, X_u^{(n)}))^2 du] \\
                + TE[\int_0^t (b(u, X_u^{(n+1)})-b(u, X_u^{(n)}))^2 du])
            \end{multlined} \\
            &\le \UB{2(C_{(2)}+T)K^2}{=:C_T}E[\int_0^t\left\lvert X_u^{(n)}-X_u^{(n-1)}\right\rvert^2 du] \\
            &\le C_TE[\int_0^t\sup_{r\in[0, u]}\left\lvert X_r^{(n)}-X_r^{(n-1)}\right\rvert^2 du] \\
            &\beq{\text{Fubini}}
            C_T\int_0^t E[\sup_{r\in[0, u]}\left\lvert X_r^{(n)}-X_r^{(n-1)}\right\rvert^2]du \\
            &= C_T\int_0^t g_n(u)du.
        \end{align}

        したがって $\forall n\ge1, \forall t\in[0, T]$ に対し
        \begin{align}
            g_n(t)
            &\le C_T\int_0^t dug_{n-1}(u) \\
            &\le C_T^2\int_0^t du_1\int_0^{u_1} du_2g_{n-2}(u_2) \\
            &\le \dotsb \\
            &\le C_T^{n-1}\int_0^t du_1\int_0^{u_1} du_2\dotsb\int_0^{u_{n-2}} du_{n-1}\UB{g_{1}(u_{n-1})}{\le C'_T} \\
            &\le C'_T\frac{(C_Tt)^{n-1}}{(n-1)!}.
        \end{align}
        
        $t=T$ ととり Chebyshevの不等式を用いると
        \begin{align}
            P(\sup_{s\in[0, T]}\left\lvert X_s^{(n)}-X_s^{(n-1)}\right\rvert>g_n(T)^{1/4})
            \le g_n(T)\cdot g_n(T)^{-1/2}
            = g_n(T)^{1/2}.
        \end{align}

        ここで $\sum_{n=1}^{\infty}g_n(T)^{1/2}<\infty$ から,Borel-Cantelliの補題より
        \begin{align}
            P(\exists n_0\ge1\text{ s.t. }\forall n\ge n_0, 
            \sup_{s\in[0, T]}\left\lvert X_s^{(n)}-X_s^{(n-1)}\right\rvert\le g_n(T)^{1/4}) = 1.
        \end{align}

        さらに $\sum_{n=1}^{\infty}g_n(T)^{1/4}<\infty$ なので
        \begin{align}
            P(\lim_{n\to\infty}\sup_{s\in[0, T]}\left\lvert X_s^{(n)}-X_s^{(n-1)}\right\rvert=0)=1
        \end{align}
        が成り立つ.
        よって逐次近似列 $(X_t^{(n)})_{t\in[0, T]}$ は a.s. に $[0, T]$ 上一様収束するので,その極限過程を $(X_t)_{t\in[0, T]}$ と定める($(X_t)_{t\in[0, T]}$ は連続な sample paths をもち, $B$ の canonical filtration に適合しているという性質を引き継ぐ).
    \end{enumerate}
\end{proof}

\end{document}
