\documentclass{jsarticle}
% \documentclass[b4paper,landscape,14pt]{jsarticle}
\title{}
\author{}
\date{
% \number\year 年 \number\month 月
}
\usepackage{fenrir_v1_4_0}
\usepackage{ethm_v1_1_0}
\mathtoolsset{showonlyrefs=true}

\begin{document}
% \maketitle
\setcounter{section}{7}
\section{Stochastic Differential Equations}
\subsection{Motivation and General Definitions}

\begin{screen}
    \begin{thm*}[Yamada-Watanabe]
        $E(\sigma, b)$ に関して weak existence, pathwise uniqueness 成立.
        このとき
        \begin{enumerate}[label=(\roman*)]
            \item
            $E(\sigma, b)$ に関する weak uniqueness 成立.
            \item
            $\forall (\Omega, \mathcal{F}, (\mathcal{F}_t), P)$: filtered prob. sp., $\forall B$: $(\mathcal{F}_t)$-BM, $\forall x\in\mathbf{R}^d, \exists! X$: $E_x(\sigma, b)$ の(一意な)strong solution.
        \end{enumerate}
    \end{thm*}
\end{screen}

この結果は Lipschitz 連続性が成り立っているとき,直接従う.

\subsection{The Lipschitz Case}

この節では,係数 $\sigma, b$ に対して以下の条件を仮定する:
\begin{screen}
    \textbf{\underline{仮定} }
    $\sigma, b$: $\mathbf{R_+}\times\mathbf{R}^d$ 上連続かつ変数 $x$ に関して Lipschitz 連続,つまり $\exists K\ge0$ s.t. $\forall t\ge0, \forall x, \forall y\in\mathbf{R}^d,$
    \begin{align}
        \left\lvert\sigma(t, x)-\sigma(t, y)\right\rvert
        &\le K\left\lvert x-y\right\rvert, \\
        \left\lvert b(t, x)-b(t, y)\right\rvert
        &\le K\left\lvert x-y\right\rvert.
    \end{align}
\end{screen}

\bigskip

\setcounter{thm}{2}

\begin{screen}
    \begin{thm}\label{thm:803}
        上の仮定の下で
        \begin{enumerate}[label=(\roman*)]
            \item
            $E(\sigma, b)$ に関して pathwise uniqueness が成立.
            \item
            $\forall (\Omega, \mathcal{F}, (\mathcal{F}_t), P)$: filtered prob. sp., $\forall B$: $(\mathcal{F}_t)$-BM, $\forall x\in\mathbf{R}^d, \exists! X$: $E_x(\sigma, b)$ の(一意な)strong solution.
        \end{enumerate}
    \end{thm}
\end{screen}

$E(\sigma, b)$ に関して 
\begin{itemize}
    \item 
    Theorem \ref{thm:803}
    $\implies $ weak existence 成立
    \item
    Theorem 8.5
    $\implies $ weak uniqueness 成立
    
    (weak existence と pathwise uniqueness が成立していることから,Yamada-Watanabe の定理を用いても示せる)
\end{itemize}

\end{document}
